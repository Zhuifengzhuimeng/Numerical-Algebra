\documentclass[withoutpreface,bwprint]{cumcmthesis}
%这里是导言区
\usepackage{url}
\usepackage{epsfig}
\usepackage{graphicx}
\usepackage{amsmath}
\usepackage{diagbox}
\usepackage{algorithm}
\usepackage{listings}
\usepackage{algpseudocode}
\usepackage{appendix}

\title{数值代数第二章上机作业}
\date{\today}
\author{Jinze Wu 1700010643}
\begin{document}
\maketitle
\section{问题1}

用算法2.5.1估计5到20阶Hilbert矩阵的$\infty$范数条件数

\begin{tabular}{|c|c|c|}
\hline
矩阵规模&无穷范数&条件数\\
\hline
5 &2.283 &6.248e+05\\
6 &2.450 &2.641e+07\\
7 &2.593&9.852e+08\\
8 &2.718&3.387e+10\\
9 &2.829 &1.100e+12\\
10 &2.929&3.421e+13\\
11 &3.020&1.027e+15\\
12 &3.103 &3.261e+16\\
13 &3.180&4.493e+17\\
14 &3.252&1.256e+19\\
15 &3.318&9.989e+17\\
16 &3.381&1.835e+18\\
17 &3.440 &1.323e+18\\
18 &3.495&9.618e+19\\
19 &3.548&3.380e+19\\
20 &3.598&3.163e+18\\
\hline
\end{tabular}

数值结果:对Hilbert矩阵A的条件数进行计算时,当A规模较小时,计算所得结果符合指数增长的趋势,但当矩阵规模达到15时计算所得A的条件数随n的增大开始震荡。

分析:当A规模较小不非常并态时,计算得到当条件数较为精确,计算所得结果有指数增长的趋势;但当矩阵规模较大时,A的三角分解已经相当不准确,计算所得的$\vert \vert A^{-1}\vert \vert_{\infty}$要比真实值小得多,计算得到的条件数也会比真实值小得多。

\section{问题2}
\begin{tabular}{|c|c|c|}
\hline
矩阵规模&实际相对误差&计算估计相对误差\\
\hline
5 &8.737930e-16 &7.401487e-16\\
6 &1.591971e-16 &5.551115e-17\\
7 &3.670094e-16 &5.773160e-16\\
8 &1.161357e-16 &1.480297e-16\\
9 &3.240888e-15 &2.410770e-15\\
10 &1.274903e-14 &1.243450e-14\\
11 &2.485871e-14 &1.302662e-14\\
12 &4.065778e-14 &2.167155e-14\\
13 &3.584752e-14 &1.582572e-14\\
14 &7.527386e-14 &3.212245e-14\\
15 &4.185389e-13 &1.452513e-13\\
16 &5.051220e-13 &3.911792e-13\\
17 &1.695057e-12 &3.324156e-13\\
18 &5.440089e-13 &4.520828e-13\\
19 &2.743271e-12 &1.540624e-12\\
20 &9.255517e-13 &4.906692e-13\\
21 &2.310420e-12 &5.914333e-13\\
22 &2.188465e-11 &1.818634e-11\\
23 &8.622616e-11 &3.626103e-11\\
24 &1.068635e-10 &2.767588e-11\\
25 &3.380843e-11 &1.453782e-11\\
26 &5.689025e-10 &1.451976e-10\\
27 &4.562095e-11 &1.266010e-11\\
28 &6.125150e-10 &2.542527e-10\\
29 &1.891001e-09 &3.357648e-10\\
30 &8.357871e-09 &3.775375e-09\\
\hline
\end{tabular}

数值结果与总结:实际相对误差与计算估计相对误差比较接近,但有时会比计算估计相对误差大,这是由于舍入误差的影响使得计算得到的残量$\vert \vert r \vert \vert_{\infty}$会比真实值小很多

注:代码可在https://github.com/Zhuifengzhuimeng/Numerical-Algebra中找到

\begin{thebibliography}{9}%宽度9
 \bibitem{bib:one}徐书方,高立,张平文著《数值线性代数》第二版
\end{thebibliography}
\end{document}