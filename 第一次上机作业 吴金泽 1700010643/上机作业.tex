\documentclass[withoutpreface,bwprint]{cumcmthesis}
%这里是导言区
\usepackage{url}
\usepackage{epsfig}
\usepackage{graphicx}
\usepackage{amsmath}
\usepackage{diagbox}
\usepackage{algorithm}
\usepackage{listings}
\usepackage{algpseudocode}
\usepackage{appendix}

\title{数值代数上机报告}
\date{\today}
\author{Jinze Wu 1700010643}
\begin{document}
\maketitle
\section{数值结果与分析}
对于84阶方程组
\begin{equation}
\begin{bmatrix}
   6 & 1 &       &         &        &   \\
   8 &  6 & 1     &      &        &   \\
     & 8 & 6     &  1       &        &    \\
     &    &\ddots   & \ddots    & \ddots   &      \\
     &   &      &   8      &  6      &   1  \\   
     &    &       &          &   8      &   6  \\   
\end{bmatrix}
\begin{bmatrix}
$$x_{1}$$\\
$$x_{2}$$\\
$$x_{3}$$\\
$$\vdots$$\\
$$x_{n-1}$$\\
$$x_{n}$$\\
\end{bmatrix}
=
\begin{bmatrix}
7\\
15\\
15\\
\vdots\\
15\\
14\\
\end{bmatrix}
\end{equation}
%$$Ax=b$$

真解为$x=[1 \dots 1 ]^{T}$ (元素全为1的$84\time1$的矩阵)

利用matlab分别编写不选主元Gauss消去法与列主元Gauss消去法的函数计算得到解,再计算所得解与真解相比误差的二范数与无穷范数

利用不选主元Gauss消去法得到的解为$x_{1}$

利用列主元Gauss消去法得到的解为$x_{2}$

计算得$\lVert x_{1}-x \rVert_{2}= 7.259*10^{8}$, $\lVert x_{2}-x \rVert_{2}=3.782*10^{-6}$

$\lVert x_{1}-x \rVert_{\infty}=5.368*10^{8}$, $\lVert x_{2}-x \rVert_{\infty}=2.797*10^{-6}$计算时间都约为0.02s

利用不选主元Gauss消去法可能会由于小主元与机器精度使计算结果误差很大,而列主元Gauss消去法可以避免这一点

在矩阵A中有小主元的比较病态的情况,应用列主元Gauss消去法代替不选主元Gauss消去法,一般来说可以大大提高计算精度,并且需要额外付出的时间代价微乎其微!


\section{matlab 程序代码}
\begin{lstlisting}[language=matlab]
function [ x ] =LUsolve(A, b)
%LUsolve
%对于方程Ax=b,利用LU分解法求解

n = size(A,1);
U=zeros(n,n);
L=zeros(n,n);
B=A;
for i=1:n-1
    B(i+1:n,i)=B(i+1:n,i)/B(i,i);
    B(i+1:n,i+1:n)=B(i+1:n,i+1:n)-B(i+1:n,i)*B(i,i+1:n);
end
for i=1:n
    U(i,:)=[zeros(1,i-1),B(i,i:n)];
    L(i,:)=[B(i,1:i-1),1,zeros(1,n-i)];
end
for i=1:n
    z(i)=b(i);
    for j=1:i-1
    z(i)=z(i)-L(i,j)*z(j);
    end
end
x = zeros(n,1);
for i=0:n-1
    x(n-i)=(z(n-i)-U(n-i,n-i+1:n)*x(n-i+1:n,1))/U(n-i,n-i);
end
end

function [ x ] =LUcolumnsolve(A, b)
%LUcolumnsolve此处显示有关此函数的摘要
%对于方程Ax=b,利用LU列主元分解法求解

n = size(A,1);
u=zeros(n,1);%记录置换矩阵P
U=zeros(n,n);
L=zeros(n,n);

B=A;
for i=1:n-1
    [a1,a2]=max(B(i:n,i));
    u(i)=a2+i-1;
    t1=B(i,:);
    B(i,:)=B(a2+i-1,:);
    B(a2+i-1,:)=t1;
    B(i+1:n,i)=B(i+1:n,i)/B(i,i);
    B(i+1:n,i+1:n)=B(i+1:n,i+1:n)-B(i+1:n,i)*B(i,i+1:n);
end%矩阵B存储了L,U的信息
for i=1:n
    U(i,:)=[zeros(1,i-1),B(i,i:n)];
    L(i,:)=[B(i,1:i-1),1,zeros(1,n-i)];
end
for i=1:n-1
    t=b(i);
    b(i)=b(u(i));
    b(u(i))=t;
end
z = zeros(n,1);
for i=1:n
    z(i)=b(i);
    for j=1:i-1
    z(i)=z(i)-L(i,j)*z(j);
    end
end
x = zeros(n,1);
for i=0:n-1
    x(n-i)=(z(n-i)-U(n-i,n-i+1:n)*x(n-i+1:n,1))/U(n-i,n-i);
end
end

b=ones(84,1)*15;
b(1)=7;
b(84)=14;
A=full(spdiags([ones(84,1)*8,ones(84,1)*6,ones(84,1)],[-1 0 1],84,84));
x1=LUsolve(A,b);
x2=LUcolumnsolve(A,b);
c=ones(84,1);
r1=norm(x1-c,2)
r2=norm(x2-c,2)
\end{lstlisting}
\end{document}